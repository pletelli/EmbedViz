% Police et Type du document
\documentclass[a4paper]{article}


% ---------------- Declaration des packages -------------------------------

%% Français
\usepackage[utf8]{inputenc}
\usepackage[T1]{fontenc}
\usepackage[french]{babel}

%% Inclure des figures
%% NO DVI :
\usepackage[pdftex]{graphicx}
\usepackage{float}

%% Format d'images
%% NO DVI :
\DeclareGraphicsExtensions{.jpg,.pdf}

%% tableaux
\usepackage{array}

\usepackage{multicol,caption}
%\usepackage[pdftex]{hyperref}
\usepackage[a4paper, margin=0.7in]{geometry}

% math
\usepackage{amssymb}

\newenvironment{Figure}
  {\par\medskip\noindent\minipage{\linewidth}}
  {\endminipage\par\medskip}

% ----------- Début du document ----------- 

\title{TX - Embed Vizualisation}
\author{Perrine Letellier}

\begin{document}
\maketitle 



Cette TX s'inscrit dans le cadre de la recherche de l'équipe DI (Décision, Image) du laboratoire Heudiasyc (CNRS, UMR 7253). L'équipe DI travaille sur l'apprentissage statistique, et vise notamment à développer des méthodes plus performantes pour le traitement et l'analyse des grandes masses de données générées collaborativement par les internautes, comme les bases de connaissances FreeBase ou Yago. Dans le cadre du projet ANR Jeune Chercheur EVEREST, une problématique de l'équipe est l'étude d'algorithmes de fouille de données qui permettent d'une part de résumer les informations contenues dans ces bases, et d'autre part de représenter ces informations de façon à pouvoir les utiliser dans des algorithmes d'apprentissage statistique. Les données considérées sont naturellement représentées sous forme de graphe, potentiellement multi-relationnel. Le sujet de la TX est de mettre en oeuvre des algorithmes de fouille de données qui permettent de représenter les entités (noeuds du graphe) sous forme vectorielle, et de développer une interface de visualisation des entités à partir de ces vecteurs. La problématique est donc de découvrir une structure sous-jacente dans une base de donnée fixée, et d'utiliser des instruments qui permettent de visualiser les résultats de l'algorithme pour résumer la base et/ou évaluer les algorithmes de fouille de données utilisées. Ainsi, en plus du travail d'implémentation, une partie importante du travail sera de de déterminer les bons outils de visualisation et les bonnes façons d'appliquer les algorithmes pour obtenir des résumés pertinents des bases de données sous forme de graphe. 

%%\begin{multicols}{2}

\section{Difficultés rencontrées}
Environnement python. Initialement je voulais travailler avec python 3, mais impossible alors d'avoir pip dans le virtualenv


\section{Mots clés - références}
\subsection{svd, collaborative filtering}

1- code svd régularisée:

http://code.google.com/p/pyrsvd/

2- base de données movielens:

http://www.grouplens.org/node/73

3- base de données Jester Jokes

http://www.ieor.berkeley.edu/~goldberg/jester-data/

pour des infos générales:

http://www.netflixprize.com/community/viewtopic.php?id=1043

www.cs.toronto.edu/~marlin/research/thesis/cfmlp.pdf
(thèse de master de Ben Marlin)

\begin{Figure}
 \centering
\begin{tabular}{|l|l|l|l|l|l|}
   \hline
   Min. & 1st. Qu. & Med. & Mean & 3rd. Qu. & Max \\
   \hline
55.0 & 113.0 & 123.0 & 123.0 & 134.0 & 176.0\\
\hline
\end{tabular}
      \captionof{table}{Résumé numérique pour le poids des enfants nés de mères non-fumeuses}
\end{Figure}

%%\begin{Figure}
%% \centering
%% \includegraphics[width=\linewidth]{boxplot_poids.jpg}
%%\end{Figure}




\section{Mise en place de l'algorithme}

Il faut determiner $\alpha$ tel que 


\section{Représentation}
Type de représentation :
\subsection{methode des plus proches voisins}
plus proches voisins par la norme euclidienne


\subsection{ACP pour plot en 2D}


\subsection{Autres à déterminer}

%%\end{multicols}


\end{document}